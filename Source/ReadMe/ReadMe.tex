\documentclass{article}
\usepackage{listings}             % Include the listings-package
\usepackage[usenames,dvipsnames,svgnames,table]{xcolor}
\usepackage{graphicx}
\usepackage{hyperref}
\usepackage{amsmath}
\usepackage{csquotes}
\definecolor{mygreen}{rgb}{0,0.6,0}
\definecolor{mygray}{rgb}{0.5,0.5,0.5}
\definecolor{mymauve}{rgb}{0.58,0,0.82}

\newcommand{\Version}{1}
    
\title{RageQuit}
\author{The\_Cowboy}
\pagestyle{headings}
\begin{document}
\maketitle
\lstset{language=Java}          % Set your language (you can change the language for each code-block optionally)
\section{Introduction}

\lstset{ %
  backgroundcolor=\color{white},   % choose the background color; you must add \usepackage{color} or    %\usepackage{xcolor}
  basicstyle=\footnotesize,        % the size of the fonts that are used for the code
  breakatwhitespace=false,         % sets if automatic breaks should only happen at whitespace
  breaklines=true,                 % sets automatic line breaking
  captionpos=b,                    % sets the caption-position to bottom
  commentstyle=\color{mygreen},    % comment style
  deletekeywords={...},            % if you want to delete keywords from the given language
  escapeinside={\%*}{*)},          % if you want to add LaTeX within your code
  extendedchars=true,              % lets you use non-ASCII characters; for 8-bits encodings only, %does not work with UTF-8
  frame=single,                    % adds a frame around the code
  keepspaces=true,                 % keeps spaces in text, useful for keeping indentation of code %(possibly needs columns=flexible)
  keywordstyle=\color{blue},       % keyword style
  language=Java,                 % the language of the code
  morekeywords={*,...},            % if you want to add more keywords to the set
  numbers=left,                    % where to put the line-numbers; possible values are (none, left, %right)
  numbersep=5pt,                   % how far the line-numbers are from the code
  numberstyle=\tiny\color{mygray}, % the style that is used for the line-numbers
  rulecolor=\color{black},         % if not set, the frame-color may be changed on line-breaks within %not-black text (e.g. comments (green here))
  showspaces=false,                % show spaces everywhere adding particular underscores; it %overrides 'showstringspaces'
  showstringspaces=false,          % underline spaces within strings only
  showtabs=false,                  % show tabs within strings adding particular underscores
  stepnumber=2,                    % the step between two line-numbers. If it's 1, each line will be %numbered
  stringstyle=\color{mymauve},     % string literal style
  tabsize=2,                       % sets default tabsize to 2 spaces
  title=\lstname                   % show the filename of files included with \lstinputlisting; also %try caption instead of title
} %optionally)

RageQuit is a UT2004 mutator to broadcast the ``Rage Quit'' message along with a dope Announcement to the players who rage quit and rejoin the server next time.  This mutator should
work with any teamgame.

\section{Installation}

\begin{itemize}
\item place the {\color{Orange}RageQuit\Version.u} and {\color{Orange}RageQuit\Version.ucl} files in the {\ttfamily System} directory.
\item for Server install, in the {\color{Purple}UT2004.ini} or {\color{Purple}Server.ini}, add the line\\

  \fbox{%
    \parbox{\textwidth}{%
        ServerPackages = RageQuit\Version\\
    }%
}\\

Note: RageQuit\Version~ should be loaded as a mutator via appropriate command line\\

   \fbox{%
    \parbox{\textwidth}{%
        ?mutator=RageQuit\Version.RageQuit
    }%
}
\\

\end{itemize} 

\section{Configuration}
\label{sec:configuration}
There are two configurable settings for RageQuit\Version.
\subsection{RageQuit.ini}
The {\color{Orange}RageQuit.ini} file is included with the package and contains the following section\\

 \fbox{%
    \parbox{\textwidth}{%
       [RageQuit1.RageQuit]\\
       KilledRageQuitSeconds=5.000000\\
       OTeamRageQuitSeconds=5.000000
    }%
} \\

\begin{itemize}
\item KilledRageQuitSeconds is the amount of time in seconds during which a player exiting the server, after being killed by enemy, is to be considered ``Rage Quitter''.
  \item OTeamRageQuitSeconds is the amount of time in seconds during which a player exiting the server, after opposite team scores, is to be considered ``Rage Quitter''.
\end{itemize}

\section{Credits}
This mutator is product of a discussion at \href{https://miasma.org/index.php?topic=1367.msg15736#msg15736}{\color{Blue}Miasma Forum}.  Thanks Defty for suggesting the ``Algorithm''.


\end{document}